\chaves{ITIL, Governan�a, Processos, Cat�logo de Servi�os, Desenho de Servi�o}

\begin{resumo}
Dentro dos �rg�os e/ou institui��es ligadas ao Governo Federal, existe uma necessidade cada vez mais latente de se organizar e ter uma TI que funcione de maneira a atender �s necessidades da institui��o e da popula��o que por ela � atendida. Esta necessidade tem sido demandada em forma de exig�ncias, n�o somente pela pr�pria institui��o, mas pelos �rg�os de controle que a regem, atrav�s de normativas e leis que obrigam as suas entidades aut�rquicas e fundacionais a instituir e institucionalizar formas de governan�a, gest�o e controle dos servi�os de TI. Dentre as formas de atendimento a estas exig�ncias existem modelos com as melhores pr�ticas que tratam de metodologias de gerenciamento eficiente dos servi�os de TI. Um dos modelos mais utilizado atualmente � o Information Technology Infrastructure Library (ITIL). Este trabalho prop�e-se a apresentar um estudo de caso para implementa��o dessas melhores pr�ticas com a utiliza��o dos processos de Gerenciamento de Cat�logo e N�vel de Servi�o da fase de Desenho de Servi�os do ITIL v3 na Equipe de Infraestrutura do Centro de Recursos Computacionais (CERCOMP), da Universidade Federal de Goi�s (UFG), a fim de desenh�-los e, se poss�vel, aperfei�o�-los, de maneira que, com o mapeamento dos processos, possamos estabelecer Acordos de N�vel de Servi�o (ANS) para a melhoria da presta��o de servi�os.
\end{resumo}
