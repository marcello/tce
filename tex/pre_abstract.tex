\keys{ITIL, Governance, Process Management, Catalog, Design of Service.}

\begin{abstract}{Design of Service - Management of Catalog and Level of Service}
Inside of public agencies as well as institutions linked to the Federal Government, there is an increasingly latent need to organize and have an IT that works in a way that meets the needs of the institution and the population that is served by it, that is, of its users. And this need has been demanded in the form of requirements, not only by the institution itself, but also by the public agencies that control it, through normatives and laws that oblige its autarchic and foundational entities to institute and institutionalize forms of governance, management and control of IT services. Among the ways to meet these requirements there are models with the best practices that deal with methodologies for efficient management of IT services. One of the most commonly used models today is the \textit{Information Technology Infrastructure Library} - ITIL\textsuperscript{\textregistered}. This work proposes to present a case study to implement these best practices with the use of the Catalog Management and Service Level processes of the ITIL Service Design phase in the Infrastructure Division of the Computer Resource Center (CERCOMP), Federal University of Goi�s (UFG), in order to design them and, if possible, improve them, so that, with process mapping, we can establish Service Level Agreements (ANS) to improve service delivery.
\end{abstract}
