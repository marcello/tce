\keys{ITIL, Governance, Process Management, Service Management, Design of Service.}
\begin{abstract}{Definition of a reference model for the elaboration and management of IT Services Catalog at UFG}
  Following a global trend, the Brazilian Government has been experiencing an intense period of digital transformation in recent years, focusing on the efficiency of public administration, the expansion of digital services and citizen satisfaction. Universities are inserted in this context and, as a result, the IT area has been increasingly demanded for excellent solutions and services. In this context, the UFG Computing Resource Center (CERCOMP) has realized the need to adopt and institutionalize a set of best practices for IT service management, especially with the use of ITIL. This paper aims to propose a reference model for the elaboration and management of the IT Service Catalog, within the scope of the UFG, including models for service specification, service level agreement and operational level agreement, as well as related processes. To this end, we used the preparation of a technical report, based on best practices available in the literature, as well as experiences and strategies employed by other Education Intitutions. The application of the reference model and its artifacts considered the context of CERCOMP and, more specifically, the services offered by the Infrastructure Team. The technical project steps were aligned with the activities developed by the CERCOMP IT Services Catalog Working Group. The products and results presented may support the Working Group in the preparation, availability and management of the CERCOMP IT Services Catalog, as well as serving as reference for other UFG departments or other Education Institutions who wish to implement it. It is understood that the availability of the Service Catalog, integrated with ITSM tool (GLPI), will contribute significantly to the improvement of the IT services offered by CERCOMP to the university community, reflecting the end user satisfaction and the achievement of established strategic and institutional goals and objectives for IT area at UFG.
\end{abstract}
